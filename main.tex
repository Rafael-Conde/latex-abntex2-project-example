\documentclass[
    hidelinks, %para evitar que apareçam caixas vermelhas ao redor de hyperlinks
    12pt,               %Tamanho da letra
    a4paper,            %Papel
    oneside,            %Não pular paginas entre os capítulos
    english,            %Para o Latex reconhecer termos em ingles
    brazil,             %Lingua mãe do documento
]{abntex2}          %Mudamos o tipo (classe) do nosso documento. agora ele obedece às normas da ABNT, e percebam a diferença total do documento!!!! tudo foi passado para português e organizado do jeito que estamos acostumados. Esse tipo, porém, precisa de uns ajustes poucos.

% \usepackage[a4paper,top=3cm,bottom=2cm,left=3cm,right=2cm]{geometry} 
\usepackage[utf8]{inputenc}         %Caracteres especiais
\usepackage[backend=biber, natbib=true]{biblatex}
\addbibresource{Bibliografia.bib} % pode-se adicionar mais de um arquivo de bibliografia
\usepackage{hyperref}
\usepackage{csquotes}
\usepackage{graphicx}               %Esse pacote é para colocar imagens!
\usepackage{import}                 %Para importar outros documentos .tex

\usepackage{fontspec} % adiciona a possibilidade de trocar a fonte padrão do TeX
\setmainfont{Arial}
% \renewcommand{\baselinestretch}{1.5}



%Documentação:
%https://ctan.math.illinois.edu/macros/latex/contrib/abntex2/doc/abntex2.pdf
%Fontes:
%https://www.overleaf.com/learn/latex/Font_typefaces

\begin{document}

\import{}{Capa.tex}                 %Importa nossa capa
% \import{}{Folha_de_Rosto.tex}       %Importa nossa folha de rosto

\tableofcontents                    %Sumário
\newpage                            %Pula uma pagina

\textual % informa o abntex2 que o texto do documento começa aqui
% começando também a numeração das páginas e a apresentação de cabeçalho e rodapé

% seu texto do documento vai aqui
% também há a possibilidade de realizar imports





% \postextual % informa o abntex2 que a parte de texto do trabalho já terminou

% \chapter{Referências}

\printbibliography[title=Referências]

\end{document}

